\section{Etapy rozwoju technologii
informacyjnych}\label{etapy-rozwoju-technologii-informacyjnych}

\begin{table}[]
\begin{tabular}{lllll}
Technologia           & Zatosowanie                            & Zalety                                            & Wyzwania                                        & Przykłady praktyczne                           \\
Chmura sieciowa       & Przechowywanie i przetwarzanie danych  & Skalowalność, dostępność zdalna                   & Bezpieczeństwo, koszty                          & AWS, Google Cloud, Microsoft Azure             \\
Sztuczna inteligencja & Analiza danych, automatyzacja procesów & Predykcja trendów automatyzacja                   & Etyka, wymagana jakość danych                   & Chatboty, systemy rekomendacyjne               \\
Internet rzeczy       & Smart home, inteligentne miasta        & Zdalne monitorowanie, oszczędzanie energii        & Prywatność interoperacyjność                    & Smart home devices, systemy monitoringu miasta \\
Blockchain            & Kryptowaluty, rejestry transakcji      & Niezmienność danych, transparentność              & Skalowalność zużycie energii                    & Bitcoin, Etherum, rejestry łańcuchowe          \\
Telemedycyna          & Zdalna opieka medyczna                 & Wygoda pacjenta, monitorowanie stanu zdrowia nta, & Prywatność pacjentów danych, wymogi regulacyjne & Konsultacje online, monitorowanie             
\end{tabular}
\end{table}

Rozwój technologii informacyjnych można podzielić na kilka kluczowych
etapów, z których każdy wnosi nowe możliwości i wyzwania:

\begin{itemize}
\tightlist
\item
  Era mechaniczna -- pierwsze urządzenia liczące, takie jak abakus,
  mechaniczne kalkulatory i maszyny Babbage'a. Był to początek
  automatyzacji obliczeń i wprowadzenie systematycznego podejścia do
  przetwarzania danych.
\item
  Era elektroniczna -- pojawienie się komputerów opartych na układach
  scalonych i tranzystorach. Rozwój systemów operacyjnych umożliwił
  efektywne zarządzanie zasobami sprzętowymi oraz tworzenie aplikacji do
  obliczeń naukowych i biznesowych.
\item
  Era sieciowa -- rozwój Internetu, sieci komputerowych i globalnej
  komunikacji, które zrewolucjonizowały wymianę informacji na świecie. W
  tym okresie powstały pierwsze przeglądarki, poczta elektroniczna i
  protokoły sieciowe.
\item
  Era mobilna -- dominacja urządzeń przenośnych, smartfonów, tabletów i
  technologii bezprzewodowych, umożliwiających dostęp do informacji w
  czasie rzeczywistym oraz rozwój aplikacji mobilnych i platform
  społecznościowych.
\item
  Era sztucznej inteligencji -- integracja systemów uczących się i
  automatyzacja procesów decyzyjnych, rozwój chatbotów, asystentów
  głosowych oraz systemów analitycznych w przedsiębiorstwach. AI wspiera
  również procesy predykcyjne, automatyczne diagnozy medyczne i
  zarządzanie ruchem miejskim.
\item
  Era Internetu Rzeczy (IoT) -- połączenie urządzeń codziennego użytku z
  Internetem, umożliwiające zbieranie danych, inteligentne zarządzanie
  środowiskiem fizycznym, monitorowanie infrastruktury miejskiej oraz
  automatyzację gospodarstw domowych.
\item
  Era komputerów kwantowych i zaawansowanych algorytmów -- wprowadzenie
  nowych paradygmatów obliczeniowych, które znacząco zwiększają moc
  przetwarzania danych i rozwiązywania złożonych problemów, takich jak
  modelowanie molekularne, optymalizacja logistyczna czy kryptografia
  postkwantowa. Każdy z tych etapów charakteryzował się nie tylko
  wzrostem możliwości technicznych, ale także koniecznością adaptacji
  społecznej, edukacyjnej i prawnej. W miarę rozwoju technologii
  powstają też nowe modele regulacyjne, standardy i normy
  bezpieczeństwa, które mają chronić użytkowników i organizacje.
\end{itemize}
