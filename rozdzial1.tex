\section{Wprowadzenie do technologii
informacyjnych}\label{wprowadzenie-do-technologii-informacyjnych}

Technologie informacyjne to zbiór narzędzi, metod i procesów, które
umożliwiają przetwarzanie, gromadzenie, przesyłanie i wykorzystywanie
informacji w sposób efektywny. Ich rozwój w ostatnich dekadach
diametralnie zmienił sposób, w jaki funkcjonują społeczeństwa,
gospodarki oraz instytucje publiczne. Współczesny świat praktycznie nie
może istnieć bez technologii informacyjnych, ponieważ są one podstawą
niemal każdego sektora życia. Rozwój technologii informacyjnych nie
tylko usprawnia komunikację i dostęp do danych, ale także kształtuje
nowe modele biznesowe, edukacyjne oraz administracyjne. W tym obszarze
powstają innowacyjne rozwiązania, które wspierają inteligentne miasta,
zrównoważony rozwój oraz poprawę jakości życia w społeczeństwach.
